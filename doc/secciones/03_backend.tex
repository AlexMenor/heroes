\chapter{Backend}

En esta sección voy a hablar de la implementación del microservicio de geolocalización.
\section{Lenguaje de programación}
He escogido usar Typescript. Es un lenguaje de tipado gradual que transpila a Javascript.
Como ventajas sobre Javascript:
\begin{itemize}
	\item Mejor DX, sobre todo gracias al autocompletado.
	\item Detección de errores precoz.
\end{itemize}
Es cierto, sin embargo, que añade más complejidad en el setup de desarrollo (y también en el de despliegue) del que hablaré más adelante.
Es más flexible que otros lenguajes tipados estrictos como Java o C\#.
Como runtime he elegido Node. Es single threaded con non blocking IO, lo cual se ajusta bien al tipo 
de peticiones que se atienden en el backend (sin cómputos extensos).  \\
Como alternativa, podría haber elegido Deno, pero tiene un ecosistema aún en crecimiento.

\section{Framework}
La API será REST, necesitamos un framework para implementar esta interfaz. \\

He decido usar express. Es un framework muy establecido y unopinionated. Esto tiene importancia en la 
arquitectura limpia que explico en la sección **REFERENCIAR**. 

\section{Setup de desarrollo}

Para tener un entorno de desarrollo replicable por cualquiera que esté interesado en trabajar en este proyecto, 
he usado Docker y Docker Compose. He creado un Dockerfile que encapsula el microservicio. 
Por último, en Docker Compose comunico el microservicio y los otros procesos (base de datos y pub/sub).


\section{Base de datos}

Necesito una base de datos que soporte un gran volumen de operaciones y permita consultas basadas en localizaciones geográficas. Es decir, en el CAP Theorem necesito la A y la P.
En esta base de datos se almacenará de manera periódica la ubicación de cada uno de los usuarios con el fin de una consulta posterior, que nos permita saber qué usuarios
están cerca de una persona que necesita ayuda. Por ello, es más que suficiente la consistencia eventual.

\subsection{Opciones}

MongoDB tiene buen soporte para geo-queries, de hecho vienen ya instaladas, pero mongo prioriza la consistencia antes que la disponibilidad. Postgres, con PostGIS, también soporta bien
este tipo de consultas, pero prioriza la consistencia a la tolerancia a la partición. \\
CouchDB, Dynamo y Cassandra son sistemas con alta disponibilidad y tolerancia a la partición. CouchDB es la mejor opción porque Dynamo es privativa y Cassandra es más adecuada para aplicaciones con más
escrituras que actualizaciones. 

Tras usar couchDB me di cuenta de que el soporte para queries de geolocalización \href{https://docs.couchdb.org/en/stable/ddocs/search.html?highlight=geospatial#geographical-searches}{no es bueno}.

Opto por tanto por MongoDB que \href{https://stackoverflow.com/questions/25734092/query-locations-within-a-radius-in-mongodb}{sí tiene buen soporte}.

Además, puedo habilitar la escritura en replicas para cambiar la consistencia por consistencia eventual y tener más disponibilidad. \href{https://stackoverflow.com/questions/11292215/where-does-mongodb-stand-in-the-cap-theorem}{Explicado aquí.}

\section{Arquitectura limpia}

He tratado de mantener en todo momento el dominio desacoplado de los detalles de implementación.

Tenemos tres capas bien diferenciadas:
\subsection{Aplicación}
En este caso es una API Rest implementada con express. Toma peticiones http, extrae la información, la valida, usa métodos expuestos del dominio y devuelve información.
Aparte, también hay un endpoint websocket que explico en **REFERENCIAR**.
\subsection{Dominio}
Tipos y procesos que exclusivamente representan el dominio.
Cuando se necesita ''materializar'' algo, se declaran interfaces. Por ejemplo, interfaz Persistencia.
\subsection{Infraestructura} 
Son servicios que implementan las interfaces de las que acabo de hablar.
En este capa está la lógica para conectarse a los distintos servicios: Bases de datos, mensajería, etc...

\section{Operaciones}

Descripción en líneas generales de los tipos de peticiones que debe atender el backend:

\subsection{Crear/Actualizar ubicación}
Toma unas coordenadas y un id de usuario y persiste en la base de datos.
Esta operación es indispensble para avisar a los usuarios cercanos a una emergencia y para 
actualizarles sobre la posición de una víctima.

\subsection{Crear alerta}
Verificando antes que el usuario no tiene ya otra alerta en curso, la persiste.
Busca todos los usuarios cercanos que hayan actualizado su ubicación recientemente y les avisa.
Mientras sigua en estado activo, se verifica que el usuario sigue actualizando su ubicación, de lo contrario la alerta pasa a estado inactivo.

\subsection{Escuchar alerta}
Los usuarios a los que se ha avisado de la alerta pueden empezar a escuchar actualizaciones de la alerta,
ya sea porque el usuario que la creó cambia su posición o porque la alerta es ya inactiva.

\subsection{Borrar alerta}
Se marca una alerta existente como inactiva. Si hay usuarios pendientes de esta alerta, se les avisa.


\section{Pub/Sub}
Al añadir websockets, teniendo en cuenta que queremos tener escalado horizontal en este servicio,
necesitamos un sistema de publicación/subscripción. \\
Para ello, he elegido REDIS, que cuenta con una interfaz muy simple. 

\section{Envío de notificaciones}
Para avisar a las personas cercanas de que alguien necesita ayuda, tenemos que tener la habilidad de enviar
notificaciones push. Para esto he usado Firebase Cloud Messaging, en la sección **REFERENCIAR** desarrollo más 
esta parte, porque tiene que ver mucho con la parte del cliente.

\section{Tests}
En este microservicio uso Jest y Supertest para testear las capas de aplicación y dominio, con mocks de la capa de infraestructura.
La arquitctura limpia facilita mucho esta tarea con inyección de dependencias.

\section{Configuración}
Las variables de configuración se definen mediante variables de entorno. Para desarrollo uso dotenv, 
que las carga de un archivo de configuración.

\section{Otras herramientas}
He utilizado express-validator para validar los cuerpos de las peticiones HTTP, eslint y prettier para 
el lint y el formato del código y nodemon para reiniciar el microservicio cada vez que hay cambios en desarrollo.