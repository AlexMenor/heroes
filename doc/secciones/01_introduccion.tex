\chapter{Introducción}

Este proyecto es software libre, y está liberado con la licencia \cite{gplv3}.

\section{El problema}
En pleno 2021, en España, la calle no es siempre un lugar seguro. Dependiendo de la zona por la que te muevas, la hora o tu género
hay más o menos probabilidades de que te lleves, en el mejor de los casos, un susto.

Los últimos estudios, como el de RACC y Zurich \cite{racc-zurich}, sacan conclusiones muy preocupantes, por ejemplo: 
\begin{itemize}
  \item El 45\% de las mujeres encuestadas declara que ha sido víctima de acoso de noche en la vía pública yendo a pie.
  \item El 68\% y 37\% de mujeres y hombres respectivamente ha cambiado alguna vez de modo de transporte por motivos de seguridad personal.
\end{itemize}

\section{Estado del arte}\label{sec:art}
Durante los últimos años, han ido emergiendo distintas aplicaciones que tratan de paliar esta inseguridad que podemos clasificar en dos grupos:
\begin{enumerate}
\item \textbf{Botón del pánico a la policía:} En caso de estar en una situación complicada, podemos avisar a la policía sin necesidad de llamar. La más famosa es \textbf{AlertCops} que cuenta con el respaldo del ministerio del interior.
\item \textbf{Botón del pánico a un grupo de contactos:} Con una activación similar a el grupo anterior, el aviso esta vez se hace a una serie de personas cercanas. En muchos casos, estas serán capaces de saber donde estás, escuchar y ver lo que pasa a partir de ese momento. Algún ejemplo: \textbf{bSafe}, \textbf{When and Where} o \textbf{Companion}.
Otra ''implementación'', es la creación de grupos de mensajería masivos de estudiantes para pedir ayuda si se sienten inseguras.
\end{enumerate}

\section{Inconvenientes}
Pese a la cantidad de alternativas y a la alta calidad de muchas de ellas, no han conseguido tener el impacto que esperaban.
\begin{enumerate}
  \item La policía o los contactos están, en muchos casos, lejos de ti y la ayuda tiene que ser inmediata.
  \item A veces el peligro no es explícito y pedir auxilio es excesivo, por ejemplo: sospechas que alguien te está siguiendo.
\end{enumerate}

\section{Análisis del problema}
Queremos reducir al máximo la inseguridad de todas las personas en la vía pública.

Podemos identificar tres niveles dentro del problema:
\begin{enumerate}
  \item \textbf{Sospecha, ligera inseguridad:} No lo sabes con certeza, pero algo te pone alerta, podrías estar en peligro y te gustaría estar acompañado.
  \item \textbf{Miedo:} Estás en una situación totalmente incómoda, no sabes si va a llegar a más, pero necesitas compañía.
  \item \textbf{Pánico:} Es necesaria una ayuda inmediata y sería pertinente avisar a los cuerpos de seguridad.
\end{enumerate}

Para los dos primeros, que a su vez son las más comunes, no hay propuestas.

\section{Objetivos generales}\label{sec:obj}
\begin{enumerate}
  \item La solución tiene que cubrir la funcionalidad de los botones del pánico existentes, analizados en la sección \ref{sec:art}.
  Dar la posibilidad de: 
  \begin{itemize}
    \item Avisar a la policía con un botón en una situación de auxilio.
    \item Avisar a un grupo de contactos de confianza con un botón, compartiendo ubicación, audio y video.
  \end{itemize}
  Configurable por el usuario.
  \item Deberá recoger toda la información relevante del entorno que pueda servir para esclarecer los hechos.
  \item La solución tiene que ser efectiva para los primeros niveles de inseguridad descritos anteriormente.
  \begin{itemize}
    \item Dar la posibilidad de que alguien que esté cerca te acompañe si lo necesitas.
    \item Dar la posiblidad de prestar esa ayuda.
  \end{itemize}
En este trabajo, vamos a plantear y a implementar una solución para este objetivo.
\end{enumerate}

