\chapter{Plataforma de desarrollo de la aplicación móvil}
Para el proyecto considero imprescindible que el código en el cliente sea de la forma de una aplicación móvil. 
Entrar a una web para el caso de uso lo descarto por dos motivos:
\begin{enumerate}
  \item Una app instalada es más accesible que entrar al navegador y después a una web.
  \item La cantidad de APIs a las que podemos acceder desde una aplicación es mayor. Y para la iniciativa de "esponja de datos" necesito acceso a APIs muy concretas que no se pueden usar a traves del navegador.
\end{enumerate}
\section{Características deseadas}

\begin{enumerate}
  \item Una base de código para iOS y Android.
  \item Paradigma declarativo.
  \item Capacidad para utilizar funcionalidades nativas y especificas del SO subyacente.
\end{enumerate}

Por la primera característica, tengo que descartar Android e IOS nativo. 

Plataformas que cumplen: 
\begin{itemize}
  \item React Native
  \item Nativescript
  \item Ionic
  \item Qasar
  \item Flutter
\end{itemize}

\section{Por qué he elegido flutter}

Si bien es cierto que React Native tiene a favor la popularidad, un lenguaje al que ya estoy habituado y más estabilidad, es más lento que Flutter y hace una traducción de componentes a código nativo.
Con Flutter sin embargo hay total control de los componentes y se muestran igual en las dos plataformas. Qasar e Ionic son muy parecidos a React Native con menos popularidad, más posibilidad de frameworks como Vue o Angular pero usando 
una webview con acceso a APIs nativas via cordova. Nativescript por último es similar a React Native, también en el método, ya que accede directamente a APIs nativas.
