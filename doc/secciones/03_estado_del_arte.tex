\chapter{Estado del arte}
\section{Algunas propuestas}
Durante los últimos años, han sido emergiendo distintas aplicaciones que tratan de paliar esta inseguridad que podemos clasificar en dos grupos:
\begin{enumerate}
\item \textbf{Botón del pánico a la policía:} En caso de estar en una situación complicada, podemos avisar a la policía sin necesidad de llamar. La más famosa es \textbf{AlertCops} que cuenta con el respaldo del ministerio del interior.
\item \textbf{Botón del pánico a un grupo de contactos:} Con una activación similar a el grupo anterior, el aviso esta vez se hace a una serie de personas cercanas. En muchos casos, estas serán capaces de saber donde estás, escuchar y ver lo que pasa a partir de ese momento. Algún ejemplo: \textbf{bSafe}, \textbf{When and Where} o \textbf{Companion}.
Otra ''implementación'', es la creación de grupos de mensajería masivos de estudiantes para pedir ayuda si se sienten inseguras.
\end{enumerate}

\section{Inconvenientes}
Pese a la cantidad de alternativas y a la alta calidad de muchas de ellas, no han conseguido tener el impacto que esperaban.
\begin{enumerate}
  \item La policía o los contactos están, en muchos casos, lejos de ti y la ayuda tiene que ser inmediata.
  \item A veces el peligro no es explícito y pedir auxilio es excesivo, por ejemplo sospechas que alguien te está siguiendo.
\end{enumerate}

