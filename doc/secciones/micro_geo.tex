\chapter{Descripción del servicio de geolocalización}

\section{Localización}
Es indispensable que los usuarios de la aplicación den permiso para obtener su localización de forma periódica (HU #6). En principio estará fijado a un intervalo de tiempo, pero sería deseable 
jugar con la variación de la ubicación para alargar o disminuir la frecuencia. Por ejemplo, si un usuario está en el trabajo, la aplicación acabaría alargando el periodo de tiempo porque la ubicación no varia apenas. 

Esta localización se almacenaría junto al id del usuario y el instante en el que se ha actualizado por última vez.

\section{Alerta}

Un usuario que tenga problemas dispararía una alerta (HU #10) (/alert POST):

\begin{enumerate}
  \item Se crea la alerta, que no es más que un registro con su id además del id del usuario que la ha creado y su estado: activa o inactiva.
  \item Entraría en un bucle mientras la alerta esté activa:
  \begin{enumerate}
    \item Busca a todos los usuarios cerca (un radio fijado) de la víctima con última actualización hace menos de un tiempo determinado.
    \item A estos usuarios les manda una notificación push (si no se les ha enviado ya para esa alerta).
  \end{enumerate}
\end{enumerate}


\section{Ayuda}

Los usuarios que reciban la notificación podrán acceder en tiempo real a la ubicación de la víctima (HU #11).

Se abriría un websocket, que mientras la alerta esté activa, enviará de forma periódica la última ubicación conocida de la persona que necesita ayuda.

Utilizo websockets para esto último porque permite al servidor hacer "push" de información sin que el cliente tenga que hacer "polling".
He considerado que podría utilizar otros protocolos como SSE (server sent events) y una implementación más peer to peer con WebRTC por ejemplo, conectando directamente los dispositivos interesados.

Es cierto que quitaría carga al servidor, pero también aumentaria la complejidad.
