\chapter{Planteamiento}

Vamos a plantear de forma general una implementación y cómo aseguramos la calidad de la misma.

\section{Metodología}

Necesito una metodología que ponga al usuario en el centro y permita una retroalimentación rápida.
Aplicando técnicas de design thinking, se ha hecho un análisis de "Personas". **INCLUIR ANEXO**
\\
Lo ideal, es que estos conceptos que expongo más abajo tengan su representación en el sistema que utilicemos
para ordenar estas tareas, en mi caso, GitHub.
\subsection{Iniciativas}
Son grandes grupos de funcionalidad, independientes entre sí, que sirven para atacar el problema.
**INSERTAR CAPTURA DE GITHUB**

\subsection{Épica}
Una iniciativa está formada por varias épicas. Una épica por si sola no es una solución al problema, pero todas las épicas de una iniciativa contribuyen a la misma solución.
**INSERTAR CAPTURA DE GITHUB**

\subsection{Historia de usuario}
Una épica a su vez está formada por varias historias de usuario que definen una funcionalidad que el usuario espera en una solución, de la forma: Como [rol] quiero [funcionalidad] para [razón].

A la hora de implementar una historia de usuario, esta se puede definir con requisitos más detallados.

**INSERTAR CAPTURA DE GITHUB**

\subsection{Issues}
Cuando no se obtiene el comportamiento esperado, podemos abrir un issue que se cerrará con un commit que lo solucione.

**INSERTAR CAPTURA DE GITHUB**

\section{Quality assurance}

Para asegurar la calidad del proyecto necesitamos utilizar integración continua con tests y análisis estático del código por medio de linters.

Cada vez que se hace un commit, se ejecutan todos estos procesos automáticamente.

La presencia de tests, de la que hablo con más detalle en **REFERENCIAR**, nos permite refactorizar con más 
tranquilidad. \\

Para cubrir la parte de integración continua he utilizado Github Actions **INSERTAR CAPTURA**, configurando en formato yaml todos los flujos.\\
La implementación continua de los tests y linters dependen del lenguaje y otros detalles de implementación, por tanto
son distintos para cada entorno.

\section{Planteamiento de implementación}

Tenemos dos partes bien diferenciadas. La parte del cliente, que se encargaría de actualizar con cierta frecuencia
la ubicación de cada usuario y recibir un aviso cuando haya alguien cercano que necesite ayuda.

Por otro lado, la parte del servidor recibe tanto las actualizaciones en cuanto a ubicación como 
los avisos de alerta. Se encarga por tanto de hacer los cálculos para saber a quién avisar y hacerlo. 
Una vez la víctima y la persona que se presta a ayudar están conectadas, el servidor debe mantener
al cliente informado sobre la localización de la víctima.