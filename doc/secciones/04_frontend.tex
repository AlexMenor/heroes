\chapter{Frontend}
\section{SDK}

Para el proyecto considero imprescindible que el código en el cliente sea de la forma de una aplicación móvil. 
Entrar a una web para el caso de uso lo descarto por dos motivos:
\begin{enumerate}
  \item Una app instalada es más accesible que entrar al navegador y después a una web.
  \item La cantidad de APIs a las que podemos acceder desde una aplicación es mayor. Y para la iniciativa de "esponja de datos" necesito acceso a APIs muy concretas que no se pueden usar a traves del navegador.
\end{enumerate}

¿Qué características son deseables?

\begin{enumerate}
  \item Una base de código para iOS y Android.
  \item Paradigma declarativo.
  \item Capacidad para utilizar funcionalidades nativas y especificas del SO subyacente.
\end{enumerate}

Por la primera característica, tengo que descartar Android e IOS nativo. 

Plataformas que cumplen: 
\begin{itemize}
  \item React Native
  \item Nativescript
  \item Ionic
  \item Qasar
  \item Flutter
\end{itemize}


Si bien es cierto que React Native tiene a favor la popularidad, un lenguaje al que ya estoy habituado y más estabilidad, es más lento que Flutter y hace una traducción de componentes a código nativo.
Con Flutter sin embargo hay total control de los componentes y se muestran igual en las dos plataformas. Qasar e Ionic son muy parecidos a React Native con menos popularidad, más posibilidad de frameworks como Vue o Angular pero usando 
una webview con acceso a APIs nativas via cordova. Nativescript por último es similar a React Native, también en el método, ya que accede directamente a APIs nativas.

\section{Mapas}

Hay dos componentes:
 \begin{itemize}
   \item Provider de mapas: Proporcionan los "tiles" que no son más que imágenes que forman el mapa. OpenStreetMap por ejemplo.
   \item Controlador: Se encarga de cargar y mostrar los tiles en la pantalla, además de reaccionar a los gestos que hace el usuario para moverse por el mapa. 
 \end{itemize}

 Estoy utilizando \href{https://wiki.openstreetmap.org/wiki/Tiles}{openstreetmap como provider}. Y como controlador \href{https://pub.dev/packages/map}{este} porque es bastante maduro, customizable y compatible con cualquier provider.

\section{Actualización de la localización}
Una parte fundamental de la aplicación es tener al día la ubicación del usuario para que sea localizable en situación de alerta.

En los siguientes escenarios:
\begin{itemize}
	\item La aplicación está en primer plano.
	\item La aplicación está abierta en segundo plano.
	\item La aplicación está cerrada.
\end{itemize}

Para ello, he usado un isolate, que se comunica con la aplicación por medio de un Port y envía la ubicación actual al servidor.
Además, usa APIs de Android que sacan partido a varios sensores del dispositivo para disparar este 
proceso solo cuando el usuario realmente está en movimiento.
También ha sido necesario registrar plugins como FCM y dotenv en java, para que estén disponibles en el isolate.

\section{Firebase Cloud Messaging}
Como he comentado en la sección **REFERENCIAR**, necesitamos un mecanismo para enviar notificaciones push a los usuarios en iOS y Android.
Además he integrar FCM en el servidor, también tenemos que hacerlo en el cliente para que las notificaciones lleguen en todos los escenarios.

\begin{itemize}
	\item La aplicación está en primer plano.
	\item La aplicación está abierta en segundo plano.
	\item La aplicación está cerrada.
\end{itemize}


Otra ventaja de usar FCM, es que nos permite obtener de manera sencilla un identificador único de dispositivo, que podemos usar en el servidor para enviar las notificaciones push.

\section{Websockets}

Para obtener actualizaciones respecto a una alerta una vez la estamos escuchando, utilizamos el endpoint **REFERENCIAR** mediante websockets.
He usado web\_socket\_channel para esto.

\section{Manejo del estado}
En flutter, como en otros frameworks, se crea un árbol de componentes.
Los datos fluyen de arriba a abajo, de padres que saben más a hijos que saben
menos. Los hijos informan a los padres de cambios mediante callbacks.
Sin embargo, hay veces que ciertos datos o comportamientos deben ser
expuestos para muchos componentes. También puede pasar que no se necesiten
en componentes superficiales y sean necesarios en otros muy profundos. \\
En un ecosistema tan grande como el de flutter, tenemos varias propuestas, más y
menos sofisticadas. Redux, BLoC/Rx o MobX son algunos ejemplos, adecuadas
para casos más pesados.\\
En mi opinión, para aplicaciones pequeñas/medianas Provider es un "sweet spot"
entre no usar estado de aplicación y usar uno muy potente. 
