\chapter{Descripción del problema}

En pleno 2020, en España, la calle no es siempre un lugar seguro. Dependiendo de la zona por la que te muevas, la hora o tu género
hay más o menos probabilidades de que te lleves, en el mejor de los casos, un susto.

Los últimos estudios, como el de RACC y Zurich \cite{racc-zurich}, sacan conclusiones muy preocupantes, por ejemplo: 
\begin{itemize}
  \item El 45\% de las mujeres encuestadas declara que ha sido víctima de acoso de noche en la vía pública yendo a pie.
  \item El 68\% y 37\% de mujeres y hombres respectivamente ha cambiado alguna vez de modo de transporte por motivos de seguridad personal.
\end{itemize}}

¿Por qué hemos normalizado está situación? ¿De verdad no podemos hacer nada al respecto? 