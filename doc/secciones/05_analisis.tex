\chapter{Análisis del problema}
 
Queremos reducir al máximo la inseguridad de todas las personas en la vía pública.
Las propuestas que hemos estudiado hasta el momento se concentran en la pedida de auxilio.

Podemos identificar tres niveles dentro del problema:
\begin{enumerate}
  \item \textbf{Sospecha, ligera inseguridad:} No lo sabes con certeza, pero algo te pone alerta, podrías estar en peligro y te gustaría estar acompañado.
  \item \textbf{Miedo:} Estás en una situación totalmente incómoda, no sabes si va a llegar a más, pero necesitas compañía.
  \item \textbf{Pánico:} Es necesaria una ayuda inmediata y sería pertinente avisar a los cuerpos de seguridad.
\end{enumerate}

Para los dos primeros, que a su vez son las mas comunes, no hay propuestas.

\section{Objetivos generales}
\begin{itemize}
  \item La solución tiene que cubrir la funcionalidad de los botones del pánico existentes, analizados en el capítulo \ref{ch:art}.
  Dar la posibilidad de: 
  \begin{itemize}
    \item Avisar a la policía con un botón en una situación de auxilio.
    \item Avisar a un grupo de contactos de confianza con un botón, compartiendo ubicación, audio y video.
  \end{itemize}
  Configurable por el usuario.
  \item La solución tiene que ser efectiva para los primeros niveles de inseguridad descritos anteriormente.
  \begin{itemize}
    \item Dar la posibilidad de que alguien que esté cerca te acompañe si lo necesitas.
    \item Dar la posiblidad de prestar esa ayuda.
  \end{itemize}
  \item Además, deberá recoger toda la información relevante del entorno que pueda servir para esclarecer los hechos.
\end{itemize}