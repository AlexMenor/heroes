\chapter{Elección de una base de datos para el microservicio de geolocalización}

\section{Características deseadas}
Necesito una base de datos que soporte un gran volumen de operaciones y permita consultas basadas en localizaciones geográficas. Es decir, en el CAP Theorem necesito la A y la P.
En esta base de datos se almacenará de manera periódica la ubicación de cada uno de los usuarios con el fin de una consulta posterior, que nos permita saber qué usuarios
están cerca de una persona que necesita ayuda. Por ello, es más que suficiente la consistencia eventual.

\section{Opciones}

MongoDB tiene buen soporte para geo-queries, de hecho vienen ya instaladas, pero mongo prioriza la consistencia antes que la disponibilidad. Postgres, con PostGIS, también soporta bien
este tipo de consultas, pero prioriza la consistencia a la tolerancia a la partición. \\
CouchDB, Dynamo y Cassandra son sistemas con alta disponibilidad y tolerancia a la partición. CouchDB es la mejor opción porque Dynamo es privativa y Cassandra es más adecuada para aplicaciones con más
escrituras que actualizacioens. 

Tras usar couchDB me di cuenta de que el soporte para queries de geolocalización \href{https://docs.couchdb.org/en/stable/ddocs/search.html?highlight=geospatial#geographical-searches}{no es bueno}.

Opto por probar ahora con MongoDB que \href{https://stackoverflow.com/questions/25734092/query-locations-within-a-radius-in-mongodb}{sí tiene buen soporte}.

Además, puedo habilitar la escritura en replicas para cambiar la consistencia por consistencia eventual y tener más disponibilidad. \href{https://stackoverflow.com/questions/11292215/where-does-mongodb-stand-in-the-cap-theorem}{Explicado aquí.}